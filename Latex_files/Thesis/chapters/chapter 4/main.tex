\chapter{Complex Langevin dynamics of spherical dimers}
\label{chapter:langevin_dynamics}
Much of the calibration theory discussed in Chapter 2 assumes that 
the target particle in question is a single sphere, one whose 
scattering and motion is easily computed. However, while working 
with dense colloidal suspensions, one often ends up trapping more 
than one sphere. Li and Arlt \cite{Li2008} studied the case of two 
microspheres trapped in a single OT and found that multiple trapped 
beads could be mistaken for a single trapped bead with altered trap 
stiffness (see \ref{sec:harmonic_traps}). Theoretical studies on the 
case of two trapped microspheres by Xu~\textit{et~al.} \cite{Xu2005} 
employed a ray-optics based model to show that the two trapped beads 
are brought into physical contact with each other by optical forces 
and they also calculated the axial equilibrium positions of the two 
trapped beads as a function of their size. Experiments in 
\cite{Praveen2016} confirmed that the two trapped beads indeed 
experience different trap stiffnesses in the vicinity of the same 
potential well.
 
There are further discussions looking into the dynamics of a whole 
host of asymmetrically shaped particles \cite{Loudet2014, ShengHua2005, Chetana2022}, their results all showing that predicting the behaviour 
an arbitrary shaped particle comes with great difficulty due to the 
fact that the optical force is dependent on a greater number of 
variables such as orientation and size factors.

With the initial goal of the PhD being to induce nucleation events 
via a spherical micro-rotor, the aim of this chapter is to - in a 
limited capacity simulate and investigate the influence of a second 
particle being bound to our target sphere. The choice of a dimer, 
instead of an amorphous solid that might better represent a growing 
crystalline solid, allows us to consider how the dynamics of the 
aggregate change by varying the size factor. Additionally, attempting 
to simulate an amorphous aggregate is rather difficult as calculating 
the optical force and torque is computationally slow. We build upon 
the works of Vigilante \textit{et al} \cite{Vigilante2020} to consider 
asymmetric spherical dimers and how varying size parameters alters the 
dynamics and additionally makes characterising their interactions within 
an optical trap more cumbersome. 

In this chapter we will consider how the addition of a second sphere 
changes the trapping dynamics by introducing multiple equilibrium 
positions. These positions are dependent on both the size ratio of 
the dimer and its orientation relative to the trap. This is important 
to address, for while the the dimers' diffusive behaviour is the same, 
regardless of position and orientation, how it interacts with the trap 
may produce the wildly different results. Furthermore, we look at how 
our dimers interact with circularly polarised light, this is especially 
pertinent as the original plan was to utilise circularly polarised 
light to generate fluid flow in a supersaturated solution. The 
rotational motion of dimers in circularly polarised light goes against 
much of the established literature and suggests that the transfer of 
spin angular momentum is enabled by combining particles together. And 
lastly, we demonstrate how a quadrant photo diode, performs in 
characterising the actual interactions between a dimer and the optical 
trap. We do so to demonstrate that much of the dimer's trajectory 
information is either lost or poorly described by simple calibration 
techniques, making their motion difficult to characterise. 

\subsection{Simulation Parameters}
As a paradigmatic example, consider a dimer suspended in water 
($n_p = 1.59, n_m = 1.33$) located at the focus of a Gaussian 
beam (more specifically a Laguerre-Gaussian beam of mode $[0,0]$, 
see \eqref{eq:incident}), the beam is focused by a objective with 
numerical aperture of 1.2 and is x polarised. The size ratio of 
the two spheres is given by $a_{I}/a_{II} = 2$ where $a_{I}$ is 
kept at $1\ \mu m$ unless specified otherwise; the dimer's 
orientation is given by a unit vector connecting the centres of 
both spheres (see \ref{sec:sim_parameters}). We define the 
'standard' orientation as being aligned with the direction of beam 
propagation direction - and therefore the 'inverted' orientation is 
defined when the dimer is orientated against the direction of beam 
orientation (see left hand side of figure~\ref{fig:paradigmatic}). 

%%%%%%%%%%%%%%%%%%%%%%%%%%%%%%%%%%%%%%%%%%%%%%%%%%%%%%%%%%%%%%%%%%%%%%%%%%%%%%%%
%%%%%%%%%%%%%%%%%%%%%%%%%%%%%%%%%%%%%%%%%%%%%%%%%%%%%%%%%%%%%%%%%%%%%%%%%%%%%%%%
\section{Positional and Orientational dependence of Trapping forces}
\label{sec:eq_positions}
If we wanted to start from first principles and determine the trap 
strength on our target particle the first step would be to locate the equilibrium position relative to the trap focus. For a single sphere 
it is easy to enough to understand that its centre of mass will be 
drawn to focal point of the laser due to gradient forces, once there 
the force is analogous to a harmonic spring with a fixed trap stiffness 
(see fig.\ref{fig:harmonic_trap}). Now, if we consider instead a dimer, 
we now have two spheres both being drawn to the focus along by the same gradient force; in addition the scattering force is significantly more 
complex due to both spheres scattering the electromagnetic fields. 
This mutual scattering between individual spheres is what makes simulating spherical aggregates far more difficult compared to a single sphere, 
and even harder still to predict the position where the dimer's centre 
of mass is in equilibrium.

Because the scattering force is only significant in the direction of 
beam propagation \cite{Capitanio2002} we can assume there is only a 
single potential well in the transverse plane. Along the beam axis, 
such an assumption is no longer valid due to fact that spherical 
aggregates experience inter-particle scattering. This is a key reason
for using \textit{mstm} as it accounts for that behaviour. The 
methodology for computing optical forces has been covered extensively 
for a number of different trapping conditions \cite{RanhaNeves2019}. 
So it is relatively easy to compute the trapping force and determine 
the equilibrium position by finding the position that minimises the 
net optical force and the local gradient is negative ($\delta F/
\delta x < 0$) - we can assume that for a dielectric sphere the 
optical torque is negligible. For a dimer (or any arbitrary spherical aggregate), we now must consider both its position and orientation 
and find where the net optical force and torque are minimised. 

After computing its $T$-matrix via \textit{mstm} and supplying that 
to \textit{ott} we compute the optical force exerted by a $50\ mW$ laser
via \eqref{eq:optical_force} in the axial direction while the dimer is 
in its 'standard' orientation. As expected we see a single point where 
the dimer will be in equilibrium, the linear fit in fig.~\ref{fig:paradigmatic}
(a) indicates a that the force can be modelled as a harmonic potential close 
to the equilibrium position ($F_z\approx-\kappa_z z$). The second point 
where the axial force goes to 0 cannot be considered as equilibrium position 
as the positive gradient indicates that the trap is unstable unless Brownian 
motion is ignored. 

We repeated the same calculation but now while the dimer is in its 'inverted' 
orientation, instead of a single point where the optical force is minimised we 
see that there are instead two separate equilibrium positions, one above the focus
and one below the focus. In this particular example the two positions are far enough
apart that both can be considered as separate harmonic traps.    
\begin{figure}[h!]
	\centering
	\begin{subfigure}{.65\linewidth}
		\includegraphics[width=\linewidth]{lam=2_theta=0.png}
		\caption{}
		\label{lam=2}
	\end{subfigure}\hfill % <-- "\hfill"
	\begin{subfigure}{.25\linewidth}
		\centering
		\raisebox{0pt}[0pt][0pt]{\makebox{}\includegraphics[width=0.3\linewidth, keepaspectratio]{theta=0.png}}
		\label{large over small}
	\end{subfigure}
	\medskip
	\begin{subfigure}{.65\linewidth}
		\includegraphics[width=\linewidth]{lam=2_theta=180.png}
		\caption{}
		\label{lam=2_inverted}
	\end{subfigure}\hfill % <-- "\hfill"
	\begin{subfigure}{.25\linewidth}
		\centering
		\raisebox{0pt}[0pt][0pt]{\makebox{}\includegraphics[width=0.3\linewidth, keepaspectratio]{theta=180.png}}
		\label{small_over_large}
	\end{subfigure}
	\caption{Plots of force vs displacement of the centre of mass of 
		the dimer (µm) for the case of a dimer of size ratio 2. (a) is 
		the case where the dimer is in its' 'standard' orientation, where
		the dimer is trapped at $z = 1.71\ \mu m$. (b) is the case where 
		the dimer is in its' 'inverted' orientation, the dimer is trapped 
		at two positions: $z = 1.20\ \mu m$ \& $z=-1.63\ \mu m$. 
		On the left are renders to visualise the dimer orientation are shown 
		below each plot. The black lines on each force-curve is a linear 
		fit with the slope being reported as the trap stiffness in the legend.}
	\label{fig:paradigmatic}
\end{figure}

We can see that both equilibrium positions have comparable axial trap 
stiffness ($\kappa_z$), however the difference in the transverse trap 
stiffness ($\kappa_x$) is far more noticeable. The same dimer was trapped
at each of the axial equilibrium positions and the transverse force was 
evaluated. While in all three cases the dimer can be trapped the linear
range where that would typically associated with a stable trap is far 
narrower in the 'standard' orientation compared to the 'inverted' cases. 
This highlights one of the challenges involved with studying 
asymmetric particles, even though its a simple enough process to trap them 
they maybe characterised very differently depending on their relative position 
and orientation towards the focus. This can have a significant impact on 
rheological studies - or attempting to probe any local property - as the 
variance in trap strength can result in large errors over repeated measurements. 
\begin{figure}[h!]
	\centering
	\includegraphics[width=\linewidth]{transverse_force.png}
	\caption{Plots of force vs displacement of the dimer's centre of mass spheres, 
		where a positive force indicates the dimer is directed right on the x-axis, 
		and vice versa for a negative force. The same simulation parameters are used here as in fig~\ref{fig:paradigmatic}(a) and (c). The blue curve representing the force response for a dimer in its standard orientation, orange being the inverted case, and green the same case but placed below the focus.}
	\label{fig:transverse_force}
\end{figure}

\newpage
For completeness the harmonic traps were located for dimers across a range 
of size ratios - from $a_{I}/a_{II} = 1$ to $a_{I}/a_{II}=10$ - while also 
recording the trap stiffness for each trap. The same simulation parameters are 
used here as for figures \ref{fig:paradigmatic} \& \ref{fig:transverse_force}. 
As shown in Fig.~\ref{fig:eq_positions} $a_{II}$ decrease the dimer begins to
approximate a single homogenous sphere - at least in terms of location and trap strength. However, for intermediate sized dimers (between $a_{I}/a_{II} = 1.1$ 
to $a_{I}/a_{II}=4$), a second equilibrium position is found below the trapping 
focus. Previous work using the ray-optics model have confirmed even in the case 
that two spheres begin separated the electric field will align the particles as 
such that they make contact and are trapped together about a single trapping 
position \cite{Xu2005}. Furthermore it has been shown through proper manipulation 
of the Gaussian or Bessel beam modes that any number of trapping potentials can 
be developed \cite{Shahabadi2020} for nanoparticles. This result however, is the 
first example of an orientation dependent trapping situation using only a 
$TEM_00$ beam. Typical experimental arrangements cannot determine much information 
on the axial position of a trapped particle relative to the trap focus; this 
result indicates not only that dimers can be trapped in multiple axial positions 
but also their trapping behaviour is heavily dependent on said axial position. 
As such it is necessary that positional information in the z-axis can be elucidated
if multiple spheres are trapped simultaneously. 
\begin{figure}[h!]
	\centering
	\begin{subfigure}{0.85\linewidth}
		\includegraphics[width=\linewidth]{Equillibrium_positions.png}
		\caption{}
		\label{eq_pos}
	\end{subfigure}\hfill % <-- "\hfill"
	\medskip
	\begin{subfigure}{0.85\linewidth}
		\includegraphics[width=\linewidth]{Equillibrium_positions_inverted.png}
		\caption{}
		\label{eq_pos_inverted}
	\end{subfigure}
	\caption{Equilibrium positions of optically trapped dimers with varying 
		size ratio, dashed lines represent unstable traps whereas solid lines 
		are for stable equilibrium positions. (a) shows that dimers while in their
		'standard' orientation will always have a single equilibrium position. 
		(b) shows that when the same dimer is in its' 'inverted' orientation 
		can be trapped in two axial positions, one below the focus and one above 
		the focus.}
	\label{fig:eq_positions}
\end{figure}

\newpage
\subsection{Non-trivial equilibrium configurations}
\label{sec:off-axis}
Computing the equilibrium positions when a dimer is aligned with the electric 
field is relatively simple as the orientational torque is minimised (see
Eq.\ref{eq:opt_torque}). Meaning once trapped the dimer is unlikely to change
orientation enough to escape the trap. Regardless, that does not rule out the 
possibility that there is a stable configuration where the orientation not 
strictly vertical, in fact most experimental work with symmetric nano-dimers 
will trap them lying perpendicular to the beam direction \cite{Ahn2018, 
Reimann2018}. Unlike in Sec.~\ref{sec:eq_positions} we cannot simply measure 
the optical force and torque as the parameter space is too large and determining 
if a particular position and orientation is stable is not clear based solely 
on force and torque measurements \cite{Bui2017}. Using the same simulation 
parameters as before we ran a number short simulations (total simulation time 
was $0.005$ s) with the laser power increased to $500\ mW$. Each simulation 
started with the dimer in a different starting position and orientation, due 
to the high laser power the dimers either escaped the trap or were stably 
trapped. The $z-\theta$ phase space - where $\theta$ is the angle between the 
direction of beam propagation and the dimer's orientation vector ($\theta=0^
\circ$ is the 'standard' orientation) - can be divided into different regions 
depending on which equilibrium configuration is reached.
\begin{figure}[h!]
	\centering
	\begin{subfigure}{0.67\linewidth}
		\includegraphics[width=\linewidth]{off_axis_trap.png}
	\end{subfigure}
	\begin{subfigure}{0.32\linewidth}
		\raisebox{0pt}[0pt][0pt]{\makebox{}\includegraphics[width=\linewidth, keepaspectratio]{off_axis_render.png}}
		\caption{}
	\end{subfigure}
	%\captionsetup{hangindent=2.35cm}
	\caption{Map of $z-\theta$ phase space using a dimer of size ratio 2 with 
		a laser power of 500 mW ($\theta=0\circ$ is the 'standard' orientation 
		and $\theta=180^\circ$ is the 'inverted' orientation). The stable 
		configurations are indicated by the larger circles and the starting 
		conditions are colour coded to match the stable point they end up in. 
		Right hand render shows a dimer in its off-axis configuration.}
	\label{fig:off_axis}
\end{figure}

Interestingly while the trap strength of these off-axis traps are similar 
in magnitude to the vertically aligned dimers, but when the laser power 
is lowered (around $5\ mW$) the traps become metastable; after reaching its'
equilibrium configuration the particle behaves similarly to a typically
trapped dimer but the trapping potential is small enough that the dimer
escapes in as little time as less than a second or after nearly a full 
3 seconds. Running similar simulations but for vertical configurations 
sees the dimer remaining trapped, even after 30 seconds of run time, 
indicating that the trapping potential is far greater than the thermal 
energy. This suggests that the reason this off-axis configuration is 
metastable is due to the increased rotational freedom. Similar 
configurations have been explored with ellipsoids; Zhu \textit{et al.} 
looked at the dynamics of various elliptical particles and found that 
regardless of shape or initial orientation the particle would tend towards
either a purely vertical or purely horizontal orientation \cite{Zhu2021}. 
It highlights an interesting aspect of the dimer's behaviour, namely that 
the potential phase space $U(\textbf{r}, \theta, \phi)$ is not continuous. 

%%%%%%%%%%%%%%%%%%%%%%%%%%%%%%%%%%%%%%%%%%%%%%%%%%%%%%%%%%%%%%%%%%%%%%%%%%%%%%%%
%%%%%%%%%%%%%%%%%%%%%%%%%%%%%%%%%%%%%%%%%%%%%%%%%%%%%%%%%%%%%%%%%%%%%%%%%%%%%%%%
\section{Continuous rotational motion due to second-order scattering}
One aspect that has yet to be covered in depth with regards to spherical 
aggregates of any construction is their interaction with circularly 
polarised light. For homogenous spheres the optical torque is regarded as
being negligible as the spin density cannot impart angular momentum while 
propagating in a homogenous medium. Dimers however, have been shown to
experience an optical torque \cite{Vigilante2020, Ahn2018, Reimann2018} 
while trapped in a circular polarised beam. In our simulations we found 
that dimers would rotate about their long axis when trapped in circularly
polarised light. In this section we want to discuss how this behaviour 
is influenced by size, position, and orientation; and furthermore, we wish
to address possible explanations for this behaviour, as none of the current
theories into optically induced rotation seem plausible.

\subsection{Polarisation Dependency on Dimer trajectory}
First we wanted to confirm that the rotation seen in our simulations is
based on the polarisation of the trapping beam. To that end, we simulated 
the motion of an optically trapped dimer in beams of varying polarisation 
($NA=1.2$, $P=100\ mW$), the dimer is composed of polystyrene ($n_p=1.59$, 
$n_m=1.33$). Each simulation was run for $1$ second ($\Delta t =10^-5$) 
and at the end we looked at the orientational time series; the dimer's 
orientation is recorded as a quaternion which can be easily converted to 
a 3-dimensional rotation matrix. By considering only the transverse 
components ($U_{1,x}$, $U_{1,y}$, $U_{2,x}$, \& $U_{2,y}$) of the rotation 
matrix and taking the Fourier transformation of their time series reveals 
the rotational frequency. The laser power is set to $100\ mW$ to avoid 
large thermal fluctuations and so that the Fourier series of the transverse 
components approximates $\delta(\omega_{rot}-f)$ - the Dirac delta function 
centred at the rotational frequency $\omega_{rot}$.

%% Example plot of rotating beam trajectory and then the same thing but 
%% Fourier transformed
\begin{equation}
	\begin{split}
		q(t) \rightarrow R(t) =& 
		\begin{pmatrix}
			U_{1,x}(t) && U_{2,x}(t) && U_{3,x}(t) \\
			U_{1,y}(t) && U_{2,y}(t) && U_{3,y}(t) \\
			U_{1,z}(t) && U_{2,z}(t) && U_{3,z}(t) 
		\end{pmatrix} \\
		\rightarrow
		&\int^\infty_{-\infty}R(t)e^{-i2\pi ft} dt = 
		\begin{pmatrix}
			\delta(\omega_{rot}-f) && \delta(\omega_{rot}-f) && \delta(f) \\
			\delta(\omega_{rot}-f) && \delta(\omega_{rot}-f) && \delta(f)\\
			\delta(f) && \delta(f) && \delta(f)
		\end{pmatrix}
	\end{split}
\end{equation}

If the rotational frequency was not immediately obvious the simulation was
repeated but over a longer simulation time. Four different size ratio of 
dimers were studied, both in their 'standard' and 'inverted' orientations. 
The results of this are displayed in Fig.~\ref{fig:rotation_vs_pol}:
\begin{figure}[h!]
	\centering
	\includegraphics[width=\linewidth]{rotation_vs_pol.png}
	\caption{Rotation frequency vs component phase difference for differently 
		sized dimers. The solid lines represent the rotation rate experienced 
		while the dimer is in its standard orientation, whereas the solid points are for the case where the orientation is inverted. Laser power $= 100\ mW$.}
	\label{fig:rotation_vs_pol}
\end{figure}

This shows us that these optical rotations are polarisation dependent 
and not merely an example of the dimer scattering light asymmetrically. 
The question then becomes, by what mechanism is the angular momentum of
the trapping beam being transferred to the dimer.

The rotation was first noted by Vigilante and co-workers who only 
considered this behaviour for a symmetric dimer \cite{Vigilante2020}.
In their work they attribute this to spin-curl effects, in which the 
curl of the spin density leads to a second order optical force that 
orbits around the beams central axis \cite{Yevick2017}. While several 
papers have demonstrated this phenomena \cite{Zhao2007,Zhao2009, Wang2010} 
it was only properly formalised by by \cite{Ruffner2012}. In which they 
showed that the seemingly random trajectory of a trapped sphere was biased
by the polarisation state of the trapping beam. While not immediately 
evident from the trajectory the helicity of the trapping beam was revealed 
by computing the particle's probability flux using.
\begin{align}
	j(r) = \frac{1}{N-1} \sum_{j=1}^{N-1}
	\frac{r_{j+1}-r_j}{\tau}\delta_{sigma_j}\left(r-\frac{r_{j+1}+r_j}{2}\right)
	\label{eq:prob_flux}
\end{align}

where $\delta_{\sigma_j}$ is the kernel of an adaptive density estimator
\cite{Silverman1986}. \eqref{eq:prob_flux} describes the direction a trapped
sphere is most likely move in given our statistical knowledge of the 
trajectories probability density function. A finite estimation of the density
function $p(r)$ is used in \cite{Ruffner2012}.
\begin{align}
	p(r) = \frac{1}{N}\sum_{j=1}^N \delta_{\sigma_j}(r-r_j)
	\label{eq:prob_density}
\end{align}

The probability flux reveals a biased motion in the trajectory of a single 
sphere (see Fig.~\ref{fig:Ruffner_Grier}). This biased motion results in a
slight orbital motion about the central axis of the trapping beam, the orbital
frequency is shown to be proportional to the polarisation state of the trapping
beam. 
\begin{figure}[h!]
	\centering
	\includegraphics[width=\linewidth]{Ruffner_Grier_2012.pdf}
	\caption{Figure reused from \cite{Ruffner2012}. (a) shows how the momentum 
		density of a Gaussian beam is twisted while using circularly polarised 
		light. The top row (figures (b)-(d)) shows a 7 sphere cluster trapped 
		in a circularly polarised beam. Due to the clusters asymmetric 
		susceptibility to polarization the cluster rotates in the $x-y$ plane.   Whereas the bottom row (figures (e) - (g)) show the similar results for a single sphere. In this instance the sphere does not rotate but instead 
		orbits the beam axis. In both instances the motion is proportional to 
		the degree of polarisation (see figures (d) and (g)) but for the single
		sphere this motion is only revealed when using \eqref{eq:prob_flux} \& \eqref{eq:prob_density}. Reused with permission from author}
	\label{fig:Ruffner_Grier}
\end{figure}

While the results from \cite{Ruffner2012} suggest that the optical rotation 
seen in asymmetric dimers is attributed to the same spin-curl forces there 
are several discrepancies that cannot be explained purely by the spin-curl 
force. Firstly, there is the question of how the spin-curl force results in
an optical torque: Comparing figures~\ref{fig:Ruffner_Grier}(c) and (e) we 
can see that the behaviour of the 7 sphere cluster rotates about the central 
axis whereas the single sphere has an only slight orbital bias in its motion. 
The authors attributed this to the fact that the cluster is scattering light 
where the spin-curl force is more substantial due to it having a wider profile. 
This would suggest that particles' whose longer axis lies in the plane perpendicular to the direction of propagation should experience any notable degree of torque. And indeed this appears to be the case for a number of experiments involving nano-dimers and ellipsoidal particles \cite{Ahn2018, Reimann2018}.

\subsection{Comparison of optical torques for Sphere and Dimers}
It does raise the question of how come our simulations show that despite the 
dimer being orientated so their long axis' align with the parallel to the 
direction of propagation they still readily rotate. You would not expect that
the addition of a single additional sphere should drastically adjust the torque
especially if said sphere is relatively small. However when we measured the 
optical torque of a single sphere and a dimer - $a_{I}/a_{II}=10$ - we found 
the exact opposite. In both cases we used the same trapping beam as used for
figure~\ref{fig:rotation_vs_pol} but with a circularly polarised beam. Both 
the sphere and dimer were rotated in the $x-z$ plane and the all three components of the optical torque were recorded.
\begin{figure}[h!]
	\centering	
	\includegraphics[width=\linewidth]{sphere_dimer_torque.png}
	\caption{Optical toque experienced by a dimer ($a_{I}/a_{II}=10$) and a 
	single isotropic sphere. Both were rotated in the $x-z$ plane and the angle 
	between $U_z$ and the beam axis gives the polar angle.The solid lines denote 
	the torque experienced by the dimer whereas the dashed lines represent the 
	torque experienced by the sphere.}
	\label{fig:sphere_dimer_torque}
\end{figure}
 
The torques about the $x$ and $y$ axis can be somewhat understood as the second 
sphere is being drawn back towards the centre of the trap by the gradient forces. The same cannot be said for the $z$component of the optical torque; while this effect could still be attributed to the spin-curl force, but it is clear that the internal scattering between the two spheres has some unintended effects. 

\subsection{Rotational frequency as a function of size ratio and orientation}
Moving on, next we want to see how the rotational speed is influenced by 
different sized dimers. Intuitively, you would expect that a larger would 
experience a greater torque and therefore rotate faster. By repeating the same kinds of simulation as used in \ref{fig:rotation_vs_pol} but for a circularly polarised beam $\phi=90\circ$ it was found that not only is the rotation rate dependent on the size of the dimer, but also on its orientation and therefore their axial position.
\begin{figure}[h!]
  \centering
  \includegraphics[width=\linewidth]{rotation_rate_vs_size.png}
  \caption{Rotation rate plotted against dimer size ratio while trapped in a
  	circularly polarised beam; a positive rotation rate indicates clockwise 
  	rotation, whereas a negative rotation rate indicates counter-clockwise 
  	rotation. The red line is for the case of a dimer in its 'standard' 
  	orientation. The blue line is for the case when the dimer is in its 
  	'inverted' orientation while trapped above the focus of the beam. And 
  	lastly the green line is for the case when the dimer is in its 
  	'inverted' orientation, but when it is trapped below the focus of the 
  	beam.}
\end{figure}

It is difficult to see from the graph, but the rotation rate never
truly goes down to zero, reaching a minimum of $2\,{\rm Hz}$, which would
imply that a second sphere of radius $200\,{\rm nm}$ is enough to
induce rotational motion. This brings into question what mechanism is
generating the optical torque. We used \textit{mstm} to look at the stokes
parameters from the scattered field from a simple plane wave incident
on a symmetric dimer, the proportion of circularly polarised light is 
minimal compared to the proportion of plane polarised light, which 
indicates that this rotational motion is not due to any inhomogeneity 
in the dimer that might impart angular momentum to the scattered beam - 
as compared to a anisotropic scatterer like vaterite. 

These results are somewhat contradictory to other work with silica dimers
\cite{Ahn2018, Debuysschere2023,Reimann2018}; previous experiments
have trapped the dimer in an orientation perpendicular to the beam
propagation direction. The rotational motion is attributed to the
dimers asymmetric geometry creating an unbalanced polarisation susceptibility
along its long axis as compared to its short axis; therefore its long
axis is aligned with the polarisation vector and can rotate
freely \cite{Ahn2018}. This however cannot be the case with our
simulations as the dimer rotates about its long axis, meaning there
cannot be an asymmetric axis to align with the beam's polarisation
vector. Furthermore, we see a non-linear increase in the rotational
speed of our dimers with size, the drag torque from the surrounding
fluid is $\propto r^3$ so the expectation is that the rotation
frequency should fall off with increasing size. This indicates that
the rotational motion is due to the shape asymmetry of the dimer and
not solely due to the beam's angular momentum. Measurement of this
photo-kinetic force is difficult to achieve due to the fact that
previous analysis was conducted in the Rayleigh regime, where the
polarizability of our dimer can be approximated as:
\begin{align}
  \bold{p}({\bf r},t)
  =
  \alpha_x E_x({\bf r},t)\hat{\bf e}_x
  + \alpha_yE_y({\bf r},t)\hat{\bf e}_y
  + \alpha_zE_z({\bf r},t)\hat{\bf e}_z
\end{align}

where the polarizability is given as a 3D vector for the three
principle Cartesian directions. In order to measure the magnitude of
second order contributions we would need to construct a dipole array
that fully captures the scattering of a dimer. Measuring the optical torque 
makes it clear that the polarizability is a contributing factor to this optical 
rotation phenomena. 


\subsection{Gyroscopic Precession using asymmetric dimers}
As mentioned in section~\ref{sec:off-axis} for specificity sized dimers there 
is the potential for non-vertical trapping orientations in which the dimer is 
still stably trapped. When a circularly polarised beam is used the dimer exhibits gyroscopic precession. As shown in fig~\ref{fig:gyro} the dimer's trajectory exhibits periodic rotation about its long axis $U_z$, precession motion as the dimer spins around the central beam axis, and nutation motion as the dimer's orientation rocks backwards and forwards. 
\begin{figure}[h]
	\centering
	\includegraphics[width=\linewidth]{gyroscopic_precession.png}
	\caption{3 second trajectory of a dimer ($a_{I}/a_{II}=2$) trapped in an 
		off axis orientation with a circularly polarised beam ($P= 100\ mW$). 
		The far left column depicts the dimer's centre of mass position with 
		time; middle two columns are the $x$, $y$, and $z$ components of the 
		vectors $u_x$ and $u_y$; last column depicts the components of the 
		vector $u_z$ which defines the dimer's orientation.}
	\label{fig:gyro}
\end{figure}

Applying a Fourier analysis to the above trajectory reveals the 3 fundamental
frequencies typically associated with precession; the $u_{z,1}$ and $x(t)$ 
series show a precession frequency of $~1.33\ Hz$ whereas the series $u_{x,1}$ 
and $u_{y,1}$ show a combined periodic signal - a rotational frequency of 
$~23\ Hz$ and a nutation frequency of $~20\ Hz$. Previous studies into 
amorphous silica nanoparticles found a linear relationship with the 
rotational frequency and the laser power, but no such relationship existed 
with the precession frequency. Our own results shows a similar linear 
relationship with vertically aligned dimers.
\begin{figure}
	\includegraphics[width=\linewidth]{RotationvsPower.png}
	\caption{Rotational frequency vs laser power for a dimer in both vertical orientations.}
\end{figure}

The linear relationship could partly be due to fact that we do not
account for the change in viscous forces with increasing laser power. 
Due to the localised heating effect 
It is far more likely that the rotational frequency reaches a maximum 
value assuming that the bulk fluid can readily absorb the laser.

This gyroscopic motion has been demonstrated previously in nanoparticles 
\cite{Zhu2021, Rashid2018, Hoang2016, Kuhn2016} but has not been observed 
for micron scale aggregates. Since the torque applied to the dimer is 
computed by evaluating the beam coefficients of the scattered field it is 
difficult to apply this result to micro-rheology experiments as one would 
need to know the exact magnitude of the optical torque ahead of time in 
order to make estimations about the local fluid viscosity. This is trivial 
for a birefringent spherical particle, less so for spherical aggregates 
whose equilibrium position and orientation are unknown. However, further 
analysis of the mechanism behind the precessive motion of off-axis dimers
may provide insights into controlling Brownian motion. An experimental work
trying to 'cool' nano-dimers by controlling the motion in all 6 degrees of 
freedom found that even while the rotation about the short axis' could be 
controlled the free rotation around the dimers' long axis resulted in an 
unpredictable torsional vibration \cite{Bang2020}. Understanding how 
rotational motion arises in the Mie-Regieme could allow researchers to 
build a robust theoretical framework to construct beam structures that 
eliminate any unwanted rotational motion from a target particle. Conversely, 
the same framework could allow for precise measurements of the optical 
torque applied to a target particle, allowing for characterisation of 
complex shaped particles' interactions with an optical trap. 

%%%%%%%%%%%%%%%%%%%%%%%%%%%%%%%%%%%%%%%%%%%%%%%%%%%%%%%%%%%%%%%%%%%%%%%%%%%%%%%
%%%%%%%%%%%%%%%%%%%%%%%%%%%%%%%%%%%%%%%%%%%%%%%%%%%%%%%%%%%%%%%%%%%%%%%%%%%%%%%
\section{Characterisation of asymmetric dimers via PSD analysis}
As discussed in \ref{sec:simulated_QPD}, one of the methods 
developed to work in conjunction with \cite{Vigilante2020} 
is a simulated quadrant photo diode for as a position 
detection system. While it is possible to extract all of 
the relevant dynamical information from a simulation, 
confirming the same behaviour in an experimental setting can 
be challenging if dealing with a non-birefringent anisotropic 
scatterer. The QPD is composed of 4 photodiodes that measure
the intensity of light incident on their surfaces. Using 
\textit{ott} we can define a region in the far-field that is 
analogues to the surface of the QPD surface. Evaluating the 
electric field across this surface gives us an approximation
of the QPD signal outputted in an experimental situation. This
does not provide a one-to-one result however as hardware errors
(i.e. internal resistance, external light sources, and vibrations)
would distort the signal somewhat. We can use this as a means 
to examine the limitations of using back-focal plane interferometry
as a means for characterising anisotropic scatterers. 

As a benchmark we start by considering a single sphere within an 
optical trap. A single polystyrene sphere suspended in water 
($a=1\mu m$, $n_p=1.59$, $n_m=1.33$) was trapped by a focused 
Gaussian beam ($NA=1.25$) using circularly polarised light. For 
the sake of time efficiency the trajectory was sampled every 10
time steps, meaning the upper bound on the power spectra is
$f_{Nyq}=f_{sample}/2=5000\ Hz$. To optimise the frequency window
we fitted the power spectra using the aliased Lorentzian 
(Eq.~\eqref{eq:aliased_lorentzian}).
\begin{figure}[h]
	\centering
	\includegraphics[width=\linewidth]{PSD_sphere.png}
	\caption{Recorded power spectra fitted to eq.~\ref{eq:alaised_lorentzian},
		 scattered points represents the blocked data ($n_b=100$). Corner 
		 frequency for the Lorentzian curves are reported in the legend.}
	\label{fig:psd_sphere}
\end{figure} 

As shown in fig.~\ref{fig:psd_sphere}, the two power spectra report 
different corner frequencies which would be indicate that the trap 
is not perfectly circular. We can use both \textit{ott} and the 
trajectory itself to derive an estimation of the trap geometry. 
The corner frequencies and corresponding trap stiffness are 
reported below:
\newpage

\begin{center}
	\captionof{table}{QPD fitting for single sphere}
	\begin{tabular}{ |c|c|c|c|c|c|c| } 
		\hline
		Fitting parameter & \multicolumn{2}{|c|}{\textit{ott} estimates} & \multicolumn{2}{|c|}{QPD fitting} & \multicolumn{2}{|c|}{Trajectory fitting}\\
		\hline
		$f_c\ [Hz]$ & 447 & 450 & \parbox{1cm}{\centering 439} & 474 
		& \parbox{1.25cm}{\centering 523} & 513 \\
		$\sigma(f_c)\ [Hz]$ & --- & --- & 9.30 & 9.65 & 8.67 & 8.61 \\
		$\kappa\ [pN/\mu m]$ & 53.05 & 53.40 & 51.96 & 56.09 & 61.94 & 60.7 \\
		\hline
		Ellipticity &
		\multicolumn{2}{|c|}{8.16 \%} &
		\multicolumn{2}{|c|}{27.17 \%} &
		\multicolumn{2}{|c|}{13.8 \%} \\
		\hline
		
	\end{tabular}
\end{center}

Where $\sigma(f_c)$ is computed from \cite{BergSoerensen2004}, where
the variance is based upon our choice of frequency boundaries
({$f_{min}:f_{max}$} = {$100\ Hz: 5000\ Hz$}). And the ellipticity 
of the beam is given by $e = (1-\kappa_y/\kappa_x)^{0.5}$ and is a 
measure of the symmetry of the beam wavefront. Its clear from these 
initial results that the QPD is more sensitive to changes along the 
y-axis than the x-axis when compared to the direct \textit{ott} 
calculations. This is somewhat reflected in the trajectory results. 
Typically, even an industrial Gaussian beam will produce an 
elliptical diffraction limit spot when heavily focused; in their 
tutorial for optimizing the PSD analysis, Berg and Sorensen reported 
a ellipticity of around 15 \% after a total calibration time of 80 
seconds \cite{BergSoerensen2004}.

The reason for the discrepancies between all 3 methods is due to what 
is actually being measured. The \textit{ott} estimates are simply 
looking at the differences in the trapping strength along the Cartesian 
axis'. If calibrated over an long enough time frame you would expect 
that the resulting power spectra would exactly mirror the \textit{ott} predictions. However over a short calibration time a QPD can predict
large discrepancies in the trap strength. There is a clear trade off 
in terms of accuracy and computation time as shorter calibration runs 
are computationally more efficient but prone to errors. This is the case
even if we know the exact positional data.  

With this in mind, let us consider a symmetric dimer that is optically
trapped by the same Gaussian beam. Not only does the dimer's equilibrium
position change but it is subjected to rotational motion due to the 
beam's angular momentum. This is reflected 

\begin{center}
	\captionof{table}{QPD fitting for symmetric dimer}
	\begin{tabular}{ |c|c|c|c|c|c|c| } 
		\hline
		Fitting parameter & \multicolumn{2}{|c|}{\textit{ott} estimate} & \multicolumn{2}{|c|}{QPD fitting} & \multicolumn{2}{|c|}{Simulation fitting} \\
		\hline
		$f_c\ [Hz]$ & 409 & 334 & \parbox{1cm}{\centering 431} & 424 
		& \parbox{1.25cm}{\centering 274} & 285 \\
		$\kappa\ [pN/\mu m]$ & 48.51 & 39.58 & 51.13 & 50.26 & 32.45 & 33.75 \\
		\hline
		Ellipticity &
		\multicolumn{2}{|c|}{42.8 \%} &
		\multicolumn{2}{|c|}{12.7 \%} & 
		\multicolumn{2}{|c|}{13.8 \%} \\
		\hline
	\end{tabular}
\end{center}

Now we see that the \textit{ott} predicts a more elliptical 
trap compared to the QPD model which says the trap is far
more symmetrical while trapping a symmetric dimer. A potential 
reason that \textit{ott} no longer expects a circular trap 
could be due to how it computes the beam shape coefficients; 
by point matching in the far field before the focus means a 
loss in accuracy for objects that trap above the focus. 
The change in the QPD estimation can be partially explained 
by the fact that rotational effects are not accounted for in 
the Lorentzian power spectra, only translational motion. 
Typically, rotational motion is only ever detected when it 
is periodic (take for example Fig.~\ref{fig:vaterite}), when 
the motion is stochastic the entire power spectra is effected 
making it near impossible to separate the translational and 
rotational contributions from a single power spectra. 

%%%%%%%%%%%%%%%%%%%%%%%%%%%%%%%%%%%%%%%%%%%%%%%%%%%%%%%%%%%%%%%%%%%%%%%%%%%%%%%
%%%%%%%%%%%%%%%%%%%%%%%%%%%%%%%%%%%%%%%%%%%%%%%%%%%%%%%%%%%%%%%%%%%%%%%%%%%%%%%
\section{Conclusions}
Considering the simplicity of a scatterer such as a dimer, one would 
assume that the dynamics of such an object would be relatively easy 
to predict. Simulations of dimers in the Mie regime show that not 
only do they have multiple positions and orientations in which they 
can be trapped but also that their interaction with circularly 
polarised light is heavily dependent on the axial position and 
trapping orientation. 

Dimer's have the potential to be used as tunable micro-rotors, being 
simple to synthesise and can be made out of any material of choice.
The rotation demonstrated by dimer's is not accurately described in
previous literature which raises questions on the interactions between
circularly polarised light and spherical aggregates. Being made up of
homogenous materials their is no change in polarisation as expected
by birefringence. Shape birefringence is also not applicable as the 
preferred orientation is perpendicular to the polarisation axis, 
meaning there can be no asymmetry in the dimer's susceptibility. 
