\chapter{Theory and methods}
\label{chapter:theory}
The two key areas of this project cover two widely different fields; 
crystallisation theory, which is covered in the previous introduction 
section, and tweezing theory. The following chapter summarises the 
working principles behind optical tweezing, the work being done with 
optical tweezers in particle classification, and how tweezing has been 
used to investigate nucleation events. 


%%%%%%%%%%%%%%%%%%%%%%%%%%%%%%%%%%%%%%%%%%%%%%%%%%%%%%%%%%%%%%%%%%%%%%%%%%%%%%%%
%%%%%%%%%%%%%%%%%%%%%%%%%%%%%%%%%%%%%%%%%%%%%%%%%%%%%%%%%%%%%%%%%%%%%%%%%%%%%%%%
\section{Working Principle}

Optical tweezing operates on the principle that light carries both
linear and angular momentum, which can be transferred to other
objects. If light is reflected or refracted by a medium part of the
light's momentum is transferred to the medium itself. While at large
scales this effect is insignificant in most cases, at micron scales
the forces imparted by individual photons begin to become significant
enough to influence its trajectory. Ashkin's work into optical
tweezing found that micron sized entities could be spatially locked by
aiming a laser with a Gaussian intensity profile directly upward at a
the cell while being suspended in a liquid medium \cite{Ashkin1970}.
This is demonstrated diagram below where we have a simple sphere
trapped by a Gaussian beam, if we consider two rays of light hitting
our sphere we see that while in the centre each ray is refracted in an
equal opposite direction to one another and so the net force imparted
is 0.  However when displaced the net force is unbalanced and points
towards the centre of the trap; this principle can be generalised to 3
degrees of freedom if we consider a focused beam instead of a
para-axial beam, typically the trap will be weaker along the beam axis
compared to the transverse trap strength.


\begin{figure}[h]
  \centering
  \includegraphics[width=0.65\linewidth]{ot_working_principal.png}
  \caption{Working principle for an optical trap, upper image shows a sphere at the centre of a trap experiences equal forces directed towards inwards. Lower image shows that when displaced 'ray a' is refracted far more than 'ray b' resulting in a net force back towards the centre.}
\end{figure}


%%%%%%%%%%%%%%%%%%%%%%%%%%%%%%%%%%%%%%%%%%%%%%%%%%%%%%%%%%%%%%%%%%%%%%%%%%%%%%%%
%%%%%%%%%%%%%%%%%%%%%%%%%%%%%%%%%%%%%%%%%%%%%%%%%%%%%%%%%%%%%%%%%%%%%%%%%%%%%%%%
\section{Electromagnetism and optical tweezing}

Proper understanding of optical tweezing requires an understanding of
how the trapped particle interacts with the trapping laser. From a 
electromagnetism perspective the laser creates a spatially and 
temporally coherent electric field that scatters light off of a 
trapped particle.  The laws governing electric and magnetic fields 
are summarised most succinctly via the Maxwell equations. The 
differential forms of which are given below \cite{Jackson_1975}:
\begin{align}
  \nabla \cdot \mathbf{E}
  &= \frac{\rho_ v}{\epsilon_0}
  \\
  \nabla \cdot \mathbf{B}
  &= 0
  \\
  \nabla \times \mathbf{E}
  &= \frac{\partial \mathbf{B}}{\partial t}
  \\
  \nabla \times \mathbf{B}
  &= {\bf j} +\frac{\partial \mathbf{E}}{\partial t}    
\end{align}
where $\rho_v$ is the free charge density, ${\bf j}_0$ is the charge
current, and $\epsilon_0$ is the permittivity of free space. 

These 4 equations describe how the electric and magnetic fields behave
and are related to one another at a microscopic level. Any discussion 
of optical trapping is underpinned by the fact that in all cases the 
Maxwell Equations must be satisfied at the macroscopic level, where 
one must account for the medium's interactions with the EM field. 
Macroscopic forms of the maxwell equations are given below \cite{Jackson_1975}:
\begin{align}
	\nabla \cdot \mathbf{D}
	&= \rho_f
	\\
	\nabla \cdot \mathbf{B}
	&= 0
	\\
	\nabla \times \mathbf{E}
	&= \frac{\partial \mathbf{B}}{\partial t}
	\\
	\nabla \times \mathbf{H}
	&= \mathbf{J}_f +\frac{\partial \mathbf{D}}{\partial t}  
\end{align}
Where $\rho_f$ and $\mathbf{J}_f$ are the charge density and current 
caused by the presence of free charges in the medium, and $\mathbf{D}$ 
and $\mathbf{H}$ are the displacement and magnetizing fields respectively. 
The latter two are defined by:
\begin{align}
	\nonumber
	\mathbf{D} &= \epsilon_0\mathbf{E}+\mathbf{P} 
	\\ 
	\nonumber
	\mathbf{H} &= \frac{1}{\mu_0}\mathbf{B}-\mathbf{M}
\end{align}
Where $\mathbf{P}$ is the polarization field and $\mathbf{M}$ is the 
magnetisation field, these fields arise due to the bound charges 
throughout the medium interacting with the EM field. 
 
The force exerted by an optical tweezer can be subdivided into the 
gradient and scattering components, for most modelling research this 
is how the force fields are reported. The gradient force is a conservative 
force brought about by the polarisation of dielectric materials, 
anything around 10\% of the trapping wavelength will be attracted along 
the gradient of the electric field to the region of greatest intensity 
(for a simple Gaussian Beam this would be at the focal point). The 
scattering force arises from the scattered field of the trapping beam 
pushing the target object away from the focal point; unlike the gradient \\
force, the scattering force is non-conservative and is loosely 
proportional to the particle's Brownian motion. The equilibrium position 
(where the mean squared displacement is minimised) is found when the 
gradient force far exceeds the scattering force, interpreting and 
calculating these forces is dependent on the ratio between the particle 
size and the trapping wavelength.


%%%%%%%%%%%%%%%%%%%%%%%%%%%%%%%%%%%%%%%%%%%%%%%%%%%%%%%%%%%%%%%%%%%%%%%%%%%%%%%%
\subsection{Ray-Optics Regime}

The Ray-Optics model is the simplest to understand, the theory models 
the laser as a collection of individual 'rays' that propagate and are 
refracted according to Snell's Law. Based on the change in direction 
momentum is transferred to the target particle; with rays closest to 
the centre of the beam having greater intensity than those rays at 
the very edge of the beam. Consider a particle struck by two rays in 
a Gaussian beam (see Fig below), one coming close to the centre, and 
the other ray coming from the edge. As each ray is refracted by the 
target sphere a force is imparted onto it, the total force imparted 
is given by:
\begin{align}
	F_i = Q_i\frac{\Delta n P_i}{c}
\end{align}
where $Q_i$ is the trapping efficiency, $\Delta n$ is the difference
in refractive indices between the solution and the target particle,
and $P_i$ is the power of the individual ray. For a beam with a
Gaussian intensity distribution $P_i$ will fall off as you move from
the centre of the beam.  The ray optics model is ideal for


%%%%%%%%%%%%%%%%%%%%%%%%%%%%%%%%%%%%%%%%%%%%%%%%%%%%%%%%%%%%%%%%%%%%%%%%%%%%%%%%
\subsection{Lornez-Mie Theory}

The Lorenz Mie theory provides an exact solution to the maxwell equations 
for the scattering caused by an isotropic sphere. The theory describes the 
scattered wave given off by a dielectric sphere when incident by a plane 
wave as a summation of partial spherical waves. For any spherical wave the 
vector fields must solve the Helmholtz wave equation given by:
\begin{align}
	\nabla^2\mathbf{E} +k^2\mathbf{E} = 0
\end{align} 

Where $k$ is the wave number of the electromagnetic radiation ($k = 2\pi/\lambda$)
 This combined with the constraints of Maxwell's Equations leaves 
 very few exact solutions apart from spherical or planar waves, 
 both of which can be converted between with relative ease. For 
 example, a plane wave electric field can be expanded into spherical 
 harmonics and likewise any spherical wave can be described as 
 combination of plane waves offset from one another. For a single 
 plane wave the incident, internal, and scattered fields are given as:
\begin{align}
	\label{eq:lornez_mie}
  {\bf E}_{\rm inc}({\bf r})
  &=
    E_0 \sum_{nm}^\infty \left[
    a_{nm}{\bf M}_{nm}^{(1)}({\bf r})
    + b_{nm}{\bf N}_{nm}^{(1)}({\bf r})\right]
  \\
  \label{eq:internal}
  {\bf E}_{\rm int}({\bf r})
  &=
    E_0 \sum_{n=1}^\infty i^n\frac{2n+1}{n(n+1)}\left[
    -id_{nm}{\bf N}_{nm}^{(1)}({\bf r}) + c_{nm}{\bf M}_{nm}^{(1)}({\bf r})
    \right]
  \\
  {\bf E}_{\rm scat}({\bf r})
  &=
    E_0 \sum_{n=1}^\infty  i^n\frac{2n+1}{n(n+1)}\left[
      -i p_{nm}{\bf N}_{nm}^{(3)}({\bf r})+q_{nm}{\bf M}_{nm}^{(3)}({\bf r})
    \right] 
    \\
	\mathbf{E}_{tot}({\bf r}) &= 
	\begin{cases}
		{\bf E}_{\rm inc}({\bf r}) + {\bf E}_{\rm scat}({\bf r}) & \text{if \textbf{r} outside}
		\\ 
		{\bf E}_{\rm int}({\bf r}) & \text{if \textbf{r} inside}
	\end{cases}
\end{align}
where $a_{nm}$, $b_{nm}$, $c_{nm}$, $d_{nm}$, $p_{nm}$, and $q_{nm}$
are the expansion coefficients of each of the fields, and $M^{(1)}_{nm}$ 
and $N^{(1)}_{nm}$ are the vector spherical harmonics. For the incident
field computing its expansion coefficients is possible via analytical
methods and are completely dependent on the beam conditions imposed by
the user. However, computing the expansion coefficients of the
internal and external fields be far more tedious depending on the
shape of the target and what properties of the scattered field are
desired --- all of which is discussed later on.  From an optical
trapping perspective the force exerted by a focused electric field can
be found by computing the Maxwell stress tensor which only requires
the total magnitude of the incident and scattered fields, essentially
computing the momentum contained in the incident beam and how much of
it has been transferred to the target particle. 

Lorenz-Mie theory can be applied to describe the scattering from any 
particle regardless of size, however as we either increase or decrease 
the size of the target particles the infinite series converges allowing 
one to ignore much of the tedium of Lorenz-Mie theory. The ray optics 
model is what is achieved when the size of the target particle far exceeds that of the laser wavelength and thus individual ray's can have independent contributions. 
In the latter case we can simplify the scattering problem by approximating 
our target as a single dipole, focusing only its interactions with the 
incident field.

%%%%%%%%%%%%%%%%%%%%%%%%%%%%%%%%%%%%%%%%%%%%%%%%%%%%%%%%%%%%%%%%%%%%%%%%%%%%%%%%
\subsection{Rayleigh Regime}

The Rayleigh approximation is the limiting approximation for describing 
a particles motion in an electromagnetic field who's wavelength is several 
times greater than the particle's size. The underlying theory is that a \\
dielectric sphere can be treated as a dipole while in the presence of the 
electromagnetic field; in which case the scattering force is given simply 
by the scattering of the induced dipole, and the gradient force is due 
to the Lorentz force \cite{Gordon1973}. The gradient forces in the 
principle Cartesian axis are described by Harada et al \cite{YasuhiroHarada1996} 
in MKS units as a restorative rectangular force field:
\begin{align}
  F_{grad,x}
  &=-\hat{x} \frac{2\pi n_2 a^3}{c}
    \left(\frac{m^2-1}{m^2+2}\right) \frac{4\tilde{x}/w_0}{1+(2\tilde{z})^2} \times I(r)
  \\
  F_{grad,y}
  &=-\hat{y} \frac{2\pi n_2 a^3}{c}
    \left(\frac{m^2-1}{m^2+2}\right) \frac{4\tilde{y}/w_0}{1+(2\tilde{z})^2} \times I(r)
  \\
  F_{grad,z}
  &=-\hat{z} \frac{2\pi n_2 a^3}{c}
    \left(\frac{m^2-1}{m^2+2}\right) \frac{4\tilde{y}/w_0}{1+(2\tilde{z})^2}
    \nonumber \times I(r)
  \\ 
  & \times \left[1-\frac{2(\tilde{x}^2+\tilde{y}^2)}{1+(2\tilde{z})^2} \right]
  \\
  \text{where:}
  \nonumber
  \\
	I(r) &= \left(\frac{2P}{\pi w_0^2}\right) \frac{1}{1+(2\tilde{z}^2)} 
	\exp \left[ - \frac{2(\tilde{x}^2+\tilde{y}^2)}{1+(2\tilde{z})^2} \right]
\end{align}

Where $m$ is the relative refractive index ($n_1/n_2$), $\omega_0$ is 
the beam waist, and $a$ is the radius of the particle. Scattering force 
however is dependent on the effective scattering cross sectional area. 
\begin{align}
  F_{\rm scat}
  &= \hat{z} \left(\frac{n_2}{2}\right) C_{pr} I(r) \\
  \text{where:} \nonumber \\
  C_{pr} &= \frac{8}{3} \pi (ka)^4 a^2 \left(\frac{m^2-1}{m^2+2}\right)^2
\end{align}

The Rayleigh regime allows for easy computation of the gradient and 
scattering forces due to the assumption that the particle is a point 
dipole, so much so that higher order scattering problems can be 
simplified by subdividing the particle into discrete dipoles (see 
Sec~\ref{sec:scattering}). However as the target particle gets larger 
this assumption fails to accurately describe the trapping force due 
to the complexity in the gradient field. For particle sizes close to 
the laser wavelength the scattered field is best described by Lorenz-Mie theory. 


%%%%%%%%%%%%%%%%%%%%%%%%%%%%%%%%%%%%%%%%%%%%%%%%%%%%%%%%%%%%%%%%%%%%%%%%%%%%%%%%
%%%%%%%%%%%%%%%%%%%%%%%%%%%%%%%%%%%%%%%%%%%%%%%%%%%%%%%%%%%%%%%%%%%%%%%%%%%%%%%%
\section{Scattering methods}
\label{sec:scattering}
There are several methods available to calculate the scattered field 
produced by a particle, while one can compute this from directly applying
Lorenz-Mie theory for complicated particle shapes and focused beams makes it 
tedious for some applications.  
%%%%%%%%%%%%%%%%%%%%%%%%%%%%%%%%%%%%%%%%%%%%%%%%%%%%%%%%%%%%%%%%%%%%%%%%%%%%%%%%
\subsection{T-matrix Method}
The $T$-matrix method was first developed by Peter Waterman with his
research into acoustic wave scattering \cite{Waterman1969}, this 
would later be extended to electromagnetic waves. Sometimes being 
referred to as the extended boundary condition method (ECBM), 
the method replaces the scatterer with a series of surface currents
over the targets surface, these currents are chosen so that the
electric field outside is identical to the original problem \cite{Wriedt1998}. 
This choice of currents negates the need to compute the internal fields
which reduces the scattering problem to a system of linear equations 
relating the incident beam coefficients to the scattered beam coefficients
the relationship between each can be summarised as:
\begin{align}
	\begin{pmatrix}
		q_{mn} \\
		p_{mn} 
	\end{pmatrix}
	= \bold{T} 
	\begin{pmatrix}
		a_{mn} \\
		b_{mn}
	\end{pmatrix}
	= \begin{pmatrix}
		T_{11} \ T_{12} \\
		T_{21} \ T_{22}
	\end{pmatrix}
	\begin{pmatrix}
		a_{mn} \\
		b_{mn}
	\end{pmatrix}
\end{align}

Where $q_{mn}$, and $p_{mn}$ are the scattering beam coefficients, and
$a_{mn}$, and $b_{mn}$ are the coefficients of the incident beam. For 
this PhD in particular we utilise an extension of the ECBM known 
as multi-sphere $T$-matrix method (\textit{mstm}), developed by Mackowski
\cite{Mackowski2011} the computational code computes the scattered field 
from each sphere within the target cluster and the incident field itself, 
this scattering between spheres converges to a final result but modern codes 
truncate the calculations to fit a desired accuracy and computational time.

The $T$-matrix method is exceptionally useful for computing the scattering 
from any arbitrary spherical aggregate. However, the $T$-matrix method by 
itself can be computationally taxing as the number of spheres increases 
\cite{Mackowski2011}. While it is possible to solve for the scattering of
each individual sphere this is only applicable for a single orientation and 
can be even slower for large aggregates \cite{Mackowski1996, Xu1995}. The 
benefit of \textit{mstm} is that the major scattering properties (scattering 
and extinction cross sections, scattering matrices, etc) can all be computed 
both for single orientations, or averaged over multiple orientations to determine
the average scattering from the target particle. 

%%%%%%%%%%%%%%%%%%%%%%%%%%%%%%%%%%%%%%%%%%%%%%%%%%%%%%%%%%%%%%%%%%%%%%%%%%%%%%%%
\subsection{Discrete Dipole Approximation}
The discrete dipole approximation (DDA) is a general method that can 
be applied to the scattering from particles of arbitrary composition
and geometry. Developed by Purcell and Pennypacker \cite{Purcell1973}, 
the DDA method approximates the particle as being constructed of dipoles.
Each dipole interacts with both the incident field and the scattered fields
from every other dipole surrounding it. The resulting scattered field is 
identical to the scattered field produced by direct integration of 
Eq.~\ref{eq:internal} throughout the full particle volume \cite{Goedecke1988}.
The integral form for the internal electric field inside a scatterer is 
given as:
\begin{align}
	\mathbf{E(r)} = \mathbf{E_{inc}(r)} + \int_{V/V_0}d^3r\mathbf{\bar{G}(r,r')}
	\chi(\mathbf{r'})\mathbf{E(r')}
\end{align}

Where $\mathbf{\bar{G}}$ is the Greens dyadic function of free space, which
defines the impulse response between two separate dipoles; and $\chi$ is the 
susceptibility of the medium, which describes the degree of polarisation of
the medium in the presence of an electric field. With the internal field 
calculated the scattered field can be computed, one of the primary advantages
of DDA over the T-matrix method is that the the composition of the target can 
be changed freely. When comparing different computational scattering methods,
the ECBM was found to be better suited for simulating the scattering of symmetric
targets as the ECBM can use the targets symmetry to speed up calculations \cite{Wriedt1998},
however when dealing with inhomogeneous media DDA is more efficient compared to
ECBM. 

%%%%%%%%%%%%%%%%%%%%%%%%%%%%%%%%%%%%%%%%%%%%%%%%%%%%%%%%%%%%%%%%%%%%%%%%%%%%%%%%
%%%%%%%%%%%%%%%%%%%%%%%%%%%%%%%%%%%%%%%%%%%%%%%%%%%%%%%%%%%%%%%%%%%%%%%%%%%%%%%%
\section{Langevin Equation}
Describing any microscopic motion requires an understanding of a molecules
diffusive behaviour, for the case of optical tweezers the most complete model 
of diffusion is the Langevin equation. Models such as the Fickian, and Einstein 
derivations are sufficient for macroscopic behaviours the Langevin equation 
better describes the microscopic characteristics of any diffusive behaviour. 
The Fickian model is not suitable for the applications of optical tweezers as 
it assumes that all diffusive motion is driven by a gradient of molecular density
\cite{Gillespie2012}, not only is this often not the case but it fails to describe 
the motion of any single molecule, only the collective behaviour. Einstein's model 
however considers the collisions experienced by individual molecules \cite{Gillespie2012a}, 
if we assume that all collisions are random and only consider the behaviour 
after a finite time step $\delta t$ then the individual collisions should 
cancel out, of course trying to understand the dynamics over a shorter period
of time is not described by Einstein's model as you would have to consider
each collision in turn \cite{Gillespie2012, Gillespie2012}. This is insufficient 
for our interests as an optical tweezers relaxation time (given by 
$\tau = \kappa/\gamma$) is often so short that we must consider the individual 
collisions experienced by any one molecule. The Langevin model of diffusion 
therefore assumes that the net force on a particular particle is described 
fully by these individual collisions \cite{Gillespie2012c}:
\begin{align}
	m\frac{dv}{dt} + \gamma_0 v + F(t) = W(t)
\end{align}

Where the first term accounts for inertial forces, the second term accounts 
for friction forces which counteract the particles current motion ($\gamma_0$ 
is the friction coefficient), and the final term accounts for the random 
Brownian force. The $F(t)$ is there for convention which accounts for any 
external forces acting on the particle. We can say that the noise term 
$W(t)$ has a Gaussian probability, with a correlation function of:
\begin{align}
	&W(t) = \sqrt{2k_BT\gamma_0}\eta(t) \\
	&\langle W_i(t)W_j(t')\rangle = 2k_BT\gamma_0\delta(t-t')
\end{align}

The Langevin model can be extrapolated to describe the diffusive behaviour 
of an overall system, but for this project we can instead consider the 
behaviour of some particle with a diffusion tensor $D$ suspended in a 
viscous fluid and spacially localised by an optical potential with trap 
strength $\kappa$. Assuming the only external force acting on our particle 
is the laser the net force should be exactly equal to force of the stochastic
collisions due to the fluids thermal energy. If we focus our analysis when 
the particle is stably trapped and assume that the trap is harmonic we can 
model the trapping force as a Hookean spring ($F(t) \approx \kappa x(t)$). 
The full Langevin equation for an optically trapped particle is therefore given as:
\begin{align}
	\label{eq:langevin}
	m\frac{\delta^2x(t)}{\delta t^2} + \gamma_0 \frac{\delta x(t)}{\delta t} + \kappa_x x(t) = \sqrt{2k_BT}\eta(t)
\end{align}

Eq.~\ref{eq:langevin} provides an accurate description of strongly trapped 
particles, however the analytical solution of the Langevin equation requires 
integration of the white noise term making it difficult to simulate the 
trajectory of a given particle \cite{Volpe2013}. Instead it is often far 
easier to solve the equation numerically and apply use the analytical 
solution to calibrate and extract information about the particle and fluid, 
and how the two interact with the optical trap.

\subsection{Finite Difference}
The Finite Difference approach involves discretizing the time and spatial 
elements in order to approximate the higher order terms. If we assume that 
$x(t)$ is differentiable to n (we can find its $n^{th}$ derivative) then 
we can use the Taylor series expansion to get:
\begin{align}
	x(t+\Delta t) = x(t)+\frac{x'(t)}{1!}\Delta t + \frac{x^2(t)}{2!}\Delta t^2+...+\frac{x^n(t)}{n!}\Delta t^n+R_n(x(t))	
\end{align}
Where $R_n(x(t))$ is the remainder term between the Taylor expansion to 
term n and the actual expression. If we limit our approach to the first 
derivative only we find that for sufficiently small values of $R_1$ the 
velocity and acceleration can be approximated as:
\begin{align}
	x'(t) &\approx \frac{x(t+\Delta t)-x(t)}{\Delta t}
	\\
	x^{''}(t) &\approx \frac{x'(t+\Delta t)-x'(t)}{\Delta t} = \frac{x(t)-2x(t+\Delta t)+x(t+2\Delta t)}{\Delta t^2}
\end{align}
By reversing the time step (i.e. use $-\Delta t$) to approximate the \\
velocity and acceleration based on the previous time steps we can 
discretize the position by taking finitely small  time steps (i.e. 
$x(t) = x_i,\ x(t-\Delta t) = x_{i-1}$). The same cannot be done for 
white noise as no information is known about $W(t)$ at any time. We 
can instead say that the velocity of a Brownian particle should 
approximate our white noise as a random walk, where at each new time 
step the position changes randomly within a given range. Constricting 
the variance to $\sqrt{2D}/\Delta t$ allows us to represent the white 
noise using the finite-difference approach as:
\begin{align}
	m\frac{x_i-2x_{i-1}+x_{i-2}}{\Delta t^2} = -\gamma\frac{x_i-x_{i-1}}{\Delta t}+\sqrt{2k_BT\gamma}\frac{w_i}{\sqrt{\Delta t}}
\end{align}
Where $w_i$ is a random real number between -1 and 1, we can say that 
it is normally distributed around 0 for simulation purposes. We can 
rearrange this for $x_i$ to approximate the Brownian motion of a 
particle (setting $x_0=0$), where the characteristic time is $\tau = m/\gamma$. 
Now in the case of an optical trap the restoration time scale is 
given by $\tau_{OT}=\kappa_x/\gamma$ which for strongly trapped particles 
is far greater than the characteristic time, therefore for simulation 
purposes we can neglect the particle's inertia which allows us to 
write the motion of an optically trapped particle as:
\begin{align}
	\label{eq:sim_langevin}
	x_i = x_{i-1} + \tau_{OT}\ x_{i-1}\Delta t + \sqrt{2D\Delta t}\ w_i
\end{align} 

This result can be generalised for a 3-dimensional description of an 
optically trapped particle, where each Cartesian direction has its 
own unique characteristic restoration time. We see from the result 
that trajectory is dependent on only a handful of factors, the trap 
stiffness $\kappa_x$, the fluid viscosity $\gamma$, and the thermal 
energy of the system $k_BT$ - with the latter two being related by 
Einstein's formulation of the diffusion coefficient $D = k_BT/\gamma$. 
Therefore by calculating these parameters to a high degree of precision 
allows one to get precise description of the forces experienced by a 
target particle, which in the past has been used for highly accurate 
force transduction \cite{BergSoerensen2004, Smith2003}.

\begin{figure}[h]
	\centering
	\includegraphics[width=0.67\linewidth]{finite_differences.png}
	\caption{Example trajectory created using Finite Differences method
			 for a $2\mu m$ diameter sphere. Trap stiffness's were estimated using \textit{ott} at $\kappa_x = \kappa_y = -100\ pN/\mu m$ and $\kappa_z = -25\ pN/\mu m$. The particle's motion can be localised around the shaded ellipsoid.}
\end{figure}

\section{Calibration Techniques}
\label{sec:calibration_techniques}
There are several approaches for calibrating and characterizing the 
optical trap, each approach has its drawbacks and benefits so each 
option should be chosen based on what elements want to be characterized. 
The basis for each of these methods stems from the analytical solution 
of the Langevin equation:
\begin{align}
	\label{eq:anylitical_lang}
	x(t) = x(0)e^{-t/\tau_{OT}}+\sqrt{2D}\int^t_0dsW_x(s)e^{-(t-s)/\tau_{OT}}
\end{align}

\subsection{Potential Well Analysis}
The Langevin equation for an optically trapped assumes that the trap 
acts similar to a Hookean spring that creates a potential well about 
its centre. Therefore a simple analysis method is to understand the 
height and width of said potential well. 

Potential analysis is a useful technique for estimating the strength 
of an optical trap; this method assumes that the force acting on the 
particle is purely conservative, an accurate presupposition if we 
ignore the motion of the particle as it enters the trap. This is 
because the scattering force is far more significant far away from 
the potential well and is negligible if the trap strength is much 
greater than the thermal fluctuations. With this in mind we can 
write the probability of finding the particle at position $x$ as:
\begin{align}
	\frac{\rho(x)}{\rho_0} = e^{-\frac{U(x)}{k_{B}T}} 
\end{align}
which therefore means we can write the potential well as:
\begin{align}
	\label{eq:potential_well}
	U(x)=-k_BT\ ln\left(\frac{\rho(x)}{\rho_0} \right)
\end{align}
Now assuming our laser acts as a Gaussian beam we should be able 
to describe the probability distribution $\rho(x)$ centred at 
some equilibrium position $x_0$:
\begin{align}
	\label{eq:prob_dist}
	\rho(x)= \sqrt{\frac{\kappa_x}{2\pi k_BT}} exp\left(-\frac{\kappa_x}{2k_BT}(x-x_{eq})^2\right)
\end{align}

By inserting eq.~\ref{eq:prob_dist} into eq.~\ref{eq:potential_well} 
we can fit the potential well in order to determine the trap strength 
$\kappa_x$, and an estimation of the equilibrium position $x_{eq}$. 
This has some limitations in that the large fluctuations can throw 
off the fit meaning a longer acquisition time is necessary to properly 
fit the potential well, making it difficult to characterise weakly 
trapped particles who may not remain trapped for long. It also 
provides no information on the particle itself (i.e. the friction 
coefficient $\gamma$ and diffusion coefficient $D$).

\subsection{Equipartition method}
The Equipartition method is by far the fastest and simplest means 
for estimating the trap strength but unlike Potential Analysis is 
limited strictly to harmonic potentials. This can be often not the 
case for highly focused beams, as the trap strength can vary due to 
polarisation differences. Simply put we can use the equipartition 
theorem to relate the potential well to the particle's thermal 
energy using eq.~\ref{eq:prob_dist}:
\begin{align}
	\left<U(x)\right> = \frac{1}{2}\kappa_x\left<(x-x_{eq})^2\right> &= \int_{-\infty}^{\infty}\rho(x)(x-x_{eq})^2 = \frac{1}{2}k_BT \\
	\implies \kappa_x &= \frac{k_BT}{\left<(x-x_{eq})^2\right>} 
\end{align}

By taking a time average over multiple trajectories to get 
$\left<x-x_{eq}\right>$ we can get a fairly accurate estimation 
of the trap strength. Because of this requires a time average of 
the particle's displacement any large errors in the position 
measurement can have knock-on effects. Likewise with the potential 
analysis route, no information is gleaned about the particle itself.

\subsection{Mean Square Displacement}
Mean square displacement (MSD) is a common means of describing 
the random motion of a given particle (or group of particles). 
This is useful information if say for example we want to understand 
reaction kinetics on the surface of a catalyst, if we know how 
far its likely to move from the surface we can tell if its likely 
to react when a catalytic site becomes available. As it pertains 
to colloids, consider a suspension of silica spheres immersed in 
a fluid undergoing Brownian motion (as described by the Langevin 
Equation) so that:
\begin{align}
	mx''(t) + \gamma x'(t) = \eta(t)
\end{align}

Where $\gamma$ is the objects friction coefficient which for spheres 
is given as $\gamma = 6\pi\eta r$, and $\eta(t)$ is a random white 
noise variable that is directly related to the thermal energy of 
the surrounding fluid. If the motion is truly random then we should 
see an average displacement of 0 regardless of how long we measure 
for. If we wish to understand the effects of a given external factor 
(i.e. an electric field or localised heating), simply looking at 
displacement will reveal nothing of value as its difficult to 
differentiate between diffusive and a biased motion. 

For each sphere we can record its position in the $x-y$ plane and 
measure its displacement from a set reference point; for example 
with an optical tweezer this could be the beam focus. We can measure 
the MSD by forming a 'window' between two points in time of the 
trajectory (i.e. $t \&  t+\tau$) and sliding this window along the 
entire trajectory length - to eliminate -ve displacements we take 
the square - we can then take the average of this series. Repeating 
over a range of time lags allows us to describe the MSD as a function 
of $\tau$:
\begin{align}
	MSD(\tau) = \left<|x(t+\tau) - x(t)|^2\right>
\end{align}

If we use eq.~\ref{eq:anylitical_lang} for an optical tweezer we can 
expand out the squared term to get an analytical expression for the 
MSD as a function of time lags:
\begin{align}
	\label{eq:MSD}
	MSD(\tau) = \left<|x(t+\tau)^2-2x(t+\tau)x(t)+x(t)^2|\right> = \frac{2k_BT}{\kappa_x}\left[1-e^{-\frac{\tau}{\tau_{OT}}}\right]
\end{align}

From this expression its evident that the mean squared displacement 
increases with larger values of $\tau$ until it reaches a maximum 
value as shown below by the dotted line.
\begin{figure}[h!]
	\centering
	\includegraphics[width=\linewidth]{MSD.png}
	\caption{Example mean squared displacement using eq.~\ref{eq:MSD}, for a $1\mu m$ sphere trapped by an optical potential well. The dotted line represents the upper limit of the sphere's displacement due to the optical trap.}
\end{figure}

The MSD plot can be subdivided into two regimes, when $\tau \gg \tau_{OT}$ 
the particle is experiencing the harmonic potential described by 
the equipartition theorem, and when $\tau \ll \tau{OT}$ the particle 
is said to be freely diffusing within the trap focus. Of course for 
a freely diffusing object the MSD will never reach a plateau value, 
comparing MSD's for different particles provides a simple visual 
indicator of the difference in trapping strength. The MSD method is 
an already very versatile analytical tool for diffusive motion, however 
it is rather slow in computing time meaning it is only really beneficial 
when a high degree of accuracy is required and shorter time resolutions 
are unavailable - such as using a quadrant photo diode instead or a high 
speed CCD.

\subsubsection{Angular Mean Square Displacement (MSAD)}
It is also possible to plot the angular MSD (MSAD) using simulative data, 
however there is yet to be a analytical expression for the MSAD of a 
freely diffusing particle. Vigilante \textit{et al} \cite{Vigilante2020} 
derived the upper limit of a dimer's MSAD along its long axis by assuming 
it was strongly trapped and so had limited angular motion, there expression 
gives:
\begin{align}
	\lim_{\tau\to\infty}\left<(\Delta u_z)^2\right> = 
	2\left[1-\frac{1}{4\beta\kappa_r} 
	\left(\frac{exp(\beta\kappa_r)-1}
	{exp(\beta\kappa_r)F(\sqrt{\beta\kappa_r})
	}\right)^2\right]
\end{align}  

Vigilante's paper expressed that they couldn't compute $MSD(\tau)$ because 
they couldn't solve the Einstein-Smoluchowski equation which describes the 
diffusion constant for dielectric particles. This would require a full 
description of a particle's electrical mobility and charge distribution 
- the latter could be achieved via a discrete dipole approximation, the 
former would be dependent on both the particle's position but relative 
orientation to the electric field.

\subsection{Power Spectrum Density (PSD)}
The power spectral density (PSD) method is by far the most versatile 
method for observing the dynamics of any object within an optical trap, 
allowing for fast calibration times while also quickly filtering out 
typical noise sources. Taking the Fourier transform of a particle's 
trajectory yields:
\begin{align}
	\hat{x}(f) = \frac{(2D)^{1/2}\hat{\eta}}{2\pi(f_c-if)}
\end{align}

where $\hat{\eta}$ is the Fourier transform of the white noise, 
where the values are exponentially distributed as opposed to being 
normally distributed in the time domain \cite{BergSoerensen2004}, 
the correlation function is given as:
\begin{align}
	\left<\hat{\eta_k}\hat{\eta_l^*}\right> = t_{msr} \delta_{k,l} \rightarrow \left< \eta^4 \right> = 2t_{msr}
\end{align} 

We can therefore ignore the white noise from our analysis by looking 
at the spectral density of $\hat{x}(f)$: 
\begin{align}
	\label{eq:lorentzian}
	S_x = \frac{\hat{x}^2}{t_{msr}} = \frac{D}{2\pi(f_c^2+f^2)}
\end{align}

We can fit the Lorentzian via a simplified geometric series 
$S_x = 1/(A+Bf_k^2)$ which allows us to compute both the diffusion 
coefficient (in arbitrary units) and the corner frequency $f_c$ 
which is directly related to the trap strength via 
$f_c = \kappa_x/(2\pi\gamma)$. Like with the analytical expression 
of the mean squared displacement we see two distinct regions, when 
$f\ll f_c$ the PSD reaches a plateau value that when converted to 
length units represents the maximum displacement the particle can 
move beyond the focus, however when $f\gg f_c$ the PSD falls off 
exponentially which denotes the particle is freely diffusing within 
the beam focus.
\begin{figure}[t]
	\centering
	\includegraphics[width=\linewidth]{PSD.png}
	\caption{Example PSD fitted using Eq.~\ref{eq:lorentzian}, power spectra is collected from an optically trapped silica sphere suspended in water. The difference in magnitude is due to the asymmetry of the quadrant photo diode having a stronger signal response in the direction of the polarisation vector. Using a correction factor (see Eq.~\ref{eq:correction_factor}) will adjust the power spectra to better describe the trap shape.}
\end{figure}

The Lorentzian relationship is only valid for frequency terms up to 
the Nyquist frequency (half of our sampling rate), this is because we 
are only taking a finite sampling of the particle's trajectory meaning 
the signal is aliased. Berg and Sorensen provide a suitable modified 
Lorentzian to account for the aliasing effects \cite{BergSoerensen2004}:
\begin{align}
	\label{eq:alaised_lorentzian}
	S_x = \frac{(\Delta x)^2\Delta t}{1+c^2-2c\cos{2\pi f_k\Delta t/N}}
\end{align}
Where N is the total number of samples taken, $\Delta x \ \& \ c$ 
have no direct physical interpretation and are defined in 
\cite{BergSoerensen2004}. Further modifications can be made to the 
power spectrum model but this is only useful when a high degree of 
accuracy is necessary. Typically power spectra are recorded using 
a Quadrant Photo Diode (QPD), which records motion in voltage units
, not in units of length. To convert between the two we have a few 
methods: Firstly if we have a mono-disperse suspension of particles, 
the laser can be scanned across the surface of our target particle; 
so long as the particle's size is known a linear conversion factor 
can be used to convert from voltage to length units. Secondly, if 
the size distribution is very wide but each particle can be accurately 
sized, then a conversion factor can be approximated by comparing the 
fitted value of the diffusion coefficient, and the reported value 
given by the Stokes-Einstein relation.
\begin{align}
	\label{eq:correction_factor}
	D_{SE} = \frac{k_BT}{\gamma_0} \Rightarrow Conversion\ Factor \ [m/V]= \sqrt{\frac{D_{SE}}{D_{fit}}}
\end{align}

With the latter method, the local fluid viscosity must be known to a 
high degree of accuracy, depending on the local heating effect this 
may be as trivial increase or it may be significant enough to drastically 
alter the characterisation. PSD analysis is often seen as the gold 
standard for calibration as it can be fine tuned to the point that 
optical forces can be computed on the order of $10^{-15} N$ 
\cite{BergSoerensen2004}, it captures all of the information acquired 
by other calibration techniques while filtering out noise and requiring 
a relatively small amount of data collected. 

%%%%%%%%%%%%%%%%%%%%%%%%%%%%%%%%%%%%%%%%%%%%%%%%%%%%%%%%%%%%%%%%%%%%%%%%%%%%%%%%
%%%%%%%%%%%%%%%%%%%%%%%%%%%%%%%%%%%%%%%%%%%%%%%%%%%%%%%%%%%%%%%%%%%%%%%%%%%%%%%%
\section{Simulation of spherical aggregates}

Later chapters cover the dynamics of spherical aggregates and anisotropic 
scatterers, these subjects are particularly difficult to characterise 
using conventional calibration techniques \cite{Li2008, Yogesha2011PreciseCO}. 
As an example consider a symmetric dimer as a paradigmatic aggregate; 
if we consider the Langevin equation for such a aggregate within an optical 
trap we have:
\begin{align}
\frac{{d}\vec{r}(t)}{{dt}} = \frac{\vec{\kappa}_x}{\gamma}\vec{r}(t) + \sqrt{2\vec{D}_x}\eta(t)
\end{align}

Where $x(t)$ is replaced with $\vec{r}(t)$ to signify that the translational 
motion is generalised to a 3 dimensional case. Except now, the dimer 
is undergoing random rotational motion in addition to its Brownian 
translational motion the first term on the right hand side is no longer 
purely a function of the dimer's position but also on its orientation. 
The rotational form of the Langevin equation for a dipole within an external 
potential is given as:
\begin{align}
  \frac{{d}\vec{u}(t)}{{dt}}
  =
  \frac{\mu}{\gamma_R}\left[\vec{u}(t)\times E(t)\right]\times \vec{u}(t)
  + \sqrt{2\vec{D}_R}\lambda(t)\times \vec{u}(t)
\end{align}

Where $\vec{u}(t)$ is the unit vector aligned along the centres of the 
two spheres, $\mu$ is its dipole moment, and $\gamma_R$ is the rotational 
friction coefficient which is given as $\gamma_R = 8\pi\eta r^3$ for a 
sphere, if the dimer is within a harmonic potential we can write the first 
term on the right hand side as $\frac{\vec{\kappa}_u}{\gamma} \times 
\vec{u}(t)$, where $\vec{\kappa}_u$ is the rotational stiffness vector. 
$\lambda(t)$ is the Brownian rotations from the surrounding fluid, 
like in the translational case the Brownian rotations are normally 
distributed and are also uncorrelated so that:
\begin{align}
  \left<\lambda(t)\lambda(t')\right> = \delta_{ij}\delta(t-t')
\end{align}

For an asymmetric scatterer whose radius is comparable to that of the 
electric field's wavelength we now have a system of simultaneous equations:
\begin{align}
	\label{eq:full_langevin}
  \frac{{d}\vec{r}(t)}{{dt}}
  &=
    \frac{\vec{\kappa}_x(\vec{u}(t))}{\gamma}\vec{r}(t) + \sqrt{2D}\eta(t)
  \\
  \frac{{d}\vec{u}(t)}{{dt}}
  &=
    \frac{\vec{\kappa}_u(\vec{r}(t))}{\gamma_R}\times \vec{u}(t)
    + \sqrt{2\vec{D}_R}\lambda(t)\times \vec{u}(t)
\end{align}

Fortunately, we do not need to solve these directly as the latter two
random variables can be easily approximated if the thermal energy of
the system is known, and the rate of change can be assumed as linear if
we take a sufficiently small time step that
$\Delta t~\ll~\kappa_x/\gamma \ \& \ \Delta t \ll
\kappa_u/\gamma_R$. In doing so we now only need to compute the
optical force and torque applied to our dimer, this has already been
compiled in a MATLAB package called \textit{Optical Tweezer Toolbox}
or \textit{ott} \cite{Nieminen2007}. In which they use the results
from \cite{Farsund1996} to compute both the optical force and torque
using the beam coefficients - the form given by
\cite{Crichton2000THEMD} is an easier form to compute:
\begin{equation}
\begin{split}
  \bold{F_z}
  &=
    -\frac{1}{4\pi k^2}\sum_{n,m} \left(\frac{m}{n(n+1)}\Im[a_{nm}b^*_{nm}-p_{nm}q^*_{nm}] \right.
  \\ 
  &+\frac{1}{n+1}\left[\frac{n(n+2)(n-m+1)(n+m+1)}{(2n+1)(2n+3)} \right]^{1/2}
  \\
  & \left.\times\Im[b_{nm}b^*_{nm}+a_{nm}a^*_{n+1m}-q_{nm}q^*_{nm}+p_{nm}p^*_{n+1m}] \right)
\end{split}
\end{equation}
\begin{equation}
\begin{split}
  \bold{T_z}
  &=
    -\frac{1}{8\pi k^3}\sum_{n,m} \left(\frac{m}{n(n+1)}[|a_{nm}|^2+|b_{nm}|^2-|p_{nm}|^2-|q_{nm}|^2] \right.
  \\ 
  &+\frac{2}{n+1}\left[\frac{n(n+2)(n-m+1)(n+m+1)}{(2n+1)(2n+3)} \right]^{1/2}
  \\
  & \left.\times\Re[b_{nm}a^*_{nm}+a_{nm}b^*_{n+1m}-p_{nm}q^*_{nm}+q_{nm}p^*_{n+1m}] \right)
\end{split}
\end{equation}

where $a_{nm}$, $b_{nm}$, $p_{nm}$, and $q_{nm}$ are the beam coefficients
of the incident and scattered fields respectively. We can get the transverse 
force and torque components in a similar form by applying a simple rotation 
transformation. In order to get the scattered beam coefficients we can use 
\textit{mstm} \cite{Mackowski2011} to calculate the T-matrix of our dimer 
(or any spherical aggregate). Vigilante et al compiled together a python 
package that combines both \textit{ott} and \textit{MSTM} to simulate the 
behaviour of spherical aggregates within a predefined optical trap. 

Throughout this PhD we use the work of Vigilante to perform a systematic 
study of the dynamics demonstrated by asymmetric dimers in both plane and 
circularly polarised light. But furthermore we expand upon their work to 
create a fully functional quadrant photo-diode to simulate the response 
from a calibration test. This builds upon the work from \cite{Rohrbach2002} 
which applied the fundamental principles of Lorenz-Mie theory to replicate 
the response signal of a QPD being used in back focal-plane interferometry.

%%%%%%%%%%%%%%%%%%%%%%%%%%%%%%%%%%%%%%%%%%%%%%%%%%%%%%%%%%%%%%%%%%%%%%%%%%%%%%%%
%%%%%%%%%%%%%%%%%%%%%%%%%%%%%%%%%%%%%%%%%%%%%%%%%%%%%%%%%%%%%%%%%%%%%%%%%%%%%%%%
\subsection{Simulated Quadrant Photodiode}
\label{sec:simulated_QPD}
In order to simulate a typical experimental set up with a QPD installed 
as a position detection system we need to evaluate the total magnitude of 
the electric field incident on the photo-diode surface. While trapping a 
micro-particle, the scattered and incident fields combine together and 
interfere with one another. These fields are collected by a condenser 
lens in the far field limit and are focused onto the QPD surface, the 
total intensity can be evaluated as:
\begin{align}
I(x,y) = \epsilon_0c\left|
\begin{bmatrix} 
	E_{i,x}(x,y)+E_{s,x}(x,y) \\ 
	E_{i,y}(x,y)+E_{s,y}(x,y) \\ 
	E_{i,z}(x,y)+E_{s,z}(x,y)
\end{bmatrix} \right|^2 \times step(NA_c-\sqrt{x^2+y^2})
\end{align}

The last term is simply a representative step term that defines the outer 
limit by which we evaluate the electric field, this is analogous to our 
condenser lens removing noise from other light sources by only accepting 
light at a specific acceptance angle defined by its numerical aperture 
$NA_c$. Depending on the relative size of our particle we can adjust 
the acceptance angle, this has very little effect on the transverse 
signals, but for axial evaluations of a trapped particle the numerical 
aperture should be tuned so that the resultant response curve has negative 
slope in order to allow for axial position detection, the method for 
finding this angle $\theta_\Theta$ is discussed in \cite{Friedrich2012}.

The incident beam is simple enough to define given our set up parameters, 
for the sake of simplicity we assume that our beam is a Laguerre-Gaussian 
beam of mode $[0.0, 0.0]$ (which is simply a pure Gaussian beam). 
\textit{Ott} uses a point matching approach to approximate the beam shape 
coefficients of the incident field by fitting it to the far field estimate, 
the beam is of the form:
\begin{align}
	E_{\rm inc}(kr)=\sum^\infty_n\sum^n_{m=-n}a_{mn}RgM_{nm}(kr)+b_{nm}RgN_{nm}(kr)
\end{align}

where $RgM_{nm}(kr)$ \& $RgN_{nm}(kr)$ are regular vector spherical wave 
functions, these differ from Eq.~\ref{eq:lornez_mie} in the fact that the 
field can either be expressed as an incoming/outgoing wave (with a 
singularity at the origin) or as a regular wave around the origin; for 
incoming/outgoing waves the wave functions use the first/second forms of 
the Hankel function respectfully. In order to compute the regular spherical 
wave at the origin we replace the Hankel function with the Bessel function 
which is simply the average of the first and second forms of the Hankel 
function, so at the origin we avoid a singularity of the EM field.  

We can if we want further restrict the incident beam by applying setting 
the truncation angle to match our microscope object, this essentially 
applies a cut off point to the In order to compute the scattering from 
the target particle \textit{ott} uses the $T$-matrix method, this is not 
essential for a simple sphere but is essential for complex shaped particles 
such as dimers. The matrix from mstm needs to be repackaged to work with the
\textit{ott} software, converting it from a column vector into a system of 
sub matrices. 

The scattered and incident fields are then combined together 
in the far field to get $I(x,y)$, the quadrant and overall signals are 
calculated via:
\begin{align}
	Q_i &= \sum_{n,m} I(x_{i,n}, y_{i,m}) \\
	S_{x} &= \frac{(Q_1+Q_2)-(Q_3+Q_4)}{\sum I_0(x,y)} \\
	S_{y} &= \frac{(Q_1+Q_3)-(Q_2+Q_4)}{\sum I_0(x,y)} \\
	S_{z} &= \frac{(Q_1+Q_2+Q_3+Q_4}{\sum I_0(x,y)}
\end{align}

Where the denominator is the total intensity on the QPD while there is 
no particle within the trap. The QPD sensitivity is dependent on both 
the polarisation of the incident beam and the displacement of the target 
sphere, this is shown in fig.~\ref{fig:totalfield}.
\begin{figure}[h!]
	\centering
	\includegraphics[width=0.85\linewidth]{fixed_polarisation.png}
	\captionsetup{margin=0.5cm}
	\caption{Total Field incident on the quadrant photo diode for a several different displacements. Top Left: $2\mu m $ diameter silica sphere placed at the focus of a  beam. Top Right: silica sphere is now displaced by $1 \mu m$ along the x-axis. Bottom Right: silica sphere is now displaced by $1 \mu m$ along the y-axis. Bottom Left: silica sphere is now displaced by $0.707 \mu m$ along the x and y axis. The trapping beam is a $TEM_{00}$ Gaussian, polarised along the along the x-axis. $S_x$, $S_y$, and $S_z$ are shown to the right of each graph, $Q_1$, $Q_2$, $Q_3$, and $Q_4$ are shown above each graph.}
	\label{fig:totalfield}
\end{figure}

Rather than displacing and rotating the particle itself, we instead rotate 
and displace the incident beam and recalculate the beam coefficients; the 
total beam is then inversely rotated in order to record the QPD signal. This
ensures that for a homogenous sphere, even if the sphere itself undergoes
rotation the QPD signal is unaffected by the change in orientation.

To confirm that our method is producing accurate results, we ran a comparison
between our simulative QPD and the results from \cite{Rohrbach2002}. Where a 
$300\ nm$ diameter sphere is scanned across the path of a focused Gaussian beam
($\lambda=1064\ nm$, $NA=1.2$), the sphere has a refractive index of 1.57 and is
suspended in water ($n_{med}=1.33$) and the condenser lens has its numerical aperture
set to 0.5 ($\theta_{max} = 30^\circ$). Scanning across all three primary axis 
produced the following response curve.
\begin{figure}[h!]
	\label{fig:Rohrbach}
	\begin{subfigure}{0.475 \linewidth}
		\subcaption{}
		\includegraphics[width=\linewidth]{QPD_axial_tests.png}
	\end{subfigure}
	\begin{subfigure}{0.475 \linewidth}
		\subcaption{}
		\includegraphics[width=\linewidth]{QPD_lat_tests.png}
	\end{subfigure}
	\caption{Comparison between QPD response signal versus work conducted by Rohrbach, single sphere ($r = 150\ nm$, $n=1.57$) is scanned by a $1064\ nm$ laser and the QPD signal recorded. Solid lines represent the signal produced by QPD using \textit{ott} and points represent the signal response collected from \cite{Rohrbach2002}.}
\end{figure}

With this any trajectory can be collected from the QPD by displacing and 
rotating the beam accordingly, later in chapters \ref{chapter:langevin_dynamics} 
and \ref{chapter:simulated_detection} we discuss how to extract information 
from the QPD's signal.
